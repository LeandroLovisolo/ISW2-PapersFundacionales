% ¿Qué es una ciencia?
%   - Definición del diccionario webster.
%
% La computación, ¿es realmente una ciencia?
%   - Las "ciencias" que insisten en llamarse ciencias en realidad no lo son.
%     Ejemplos: ciencias sociales, ciencias económicas, ciencias políticas, y
%     claro, ciencias de la computación.
%   - La principal distinción entre ciencias e ingenierías: el propósito
%   - Las ciencias construyen para estudiar, mientras que las ingenierías
%     estudian para construir.
%
% ¿Cuál es nuestra disciplina?
%   - Tesis: nuestra disciplina no es una ciencia, sino una disciplina sintética;
%     una ingeniería.
%   - Nos preocupamos por *construir cosas*, sean estas computadoras, algoritmos
%     o sistemas de software.
%
% El computador científico como fabricante de herramientas
%   - A diferencia de otras disciplinas como la mecánica o la ingeniería civil,
%     las cosas que construimos no son un fin en sí mismo sino herramientas para
%     ayudar a otras personas a construir cosas que enriquecen la vida humana.
%   - Por lo tanto la visión del computador científico como fabricante de
%     herramientas es más honesta y adecuada.
%   - Criterio de éxito más adecuado: o bien las herramientas que construimos
%     ayudan a los demás a hacer su vida más fácil, o no.
%
% ¿Cómo puede engañarnos un nombre?
%   - Llamar a nuestra disciplina una ciencia es una táctica dudosa: refuerza la
%     idea que estamos por encima de las ingenierías, y por lo tanto, del objetivo
%     de construir cosas.
%   - Llamar a nuestra disciplina una ciencia es una táctica riesgosa: como en el
%     caso de las ciencias sociales, llamar a nuestra disciplina una ciencia la
%     ridiculiza y hace ver a quien la practica como un tonto.
%   - Llamar a nuestra disciplina una ciencia es inútil: debemos ser respetados
%     por nuestros logros, no por nuestro título.
%
% La frivolidad de la ciencia
%   - Las ciencias se preocupan por descubrir nuevos hechos, y enunciar y
%     publicar nuevas leyes.
%   - Si nos confundimos con científicos, corremos el riesgo de inventar nuevos
%     algoritmos, lenguajes y computadoras sólo para publicar.
%   - Pero en el mundo real, la novedad no es mérito. Sólo la utilidad y el costo
%     determinan el mérito de un invento.
%   - Al glorificar los aspectos más abstractos y "científicos" de nuestra
%     disciplina, nos olvidamos de quienes van a usar nuestros inventos.
%
% El fabricante de herramientas como colaborador
%   - Nos ayuda a mantener el foco en problemas reales, no sólo ejercicios o
%     problemas de juguete.
%   - Nos mantiene honestos en cuanto a nuestros éxitos y fracasos.
%   - Nos obliga a enfrentarnos contra la totalidad de un problema, no sólo las
%     partes "fáciles" o puramente matemáticas.
%   - Atacar problemas enteros nos obliga a desarrollar más la computación y
%     explorar áreas de la misma que de otra forma nunca consideraríamos.
%
% Conclusiones

\documentclass[spanish]{beamer}
\usepackage[utf8]{inputenc}
\usepackage[spanish, es-ucroman, es-noquoting]{babel}

%%%%%%%%%%%%%%%%%%%%%%%%%%%%%%%%%%%%%%%%%%%%%%%%%%%%%%%%%%%%%%%%%%%%%%%%%%%%%%%%
% Modo handout (comentar para versión presentación en pantalla/proyector)
% \usepackage{pgfpages}
% \pgfpagesuselayout{4 on 1}[a4paper, landscape, border shrink=5mm]
% \setbeamertemplate{background canvas}{
%     \tikz \draw (current page.north west) rectangle (current page.south east);
% }
%%%%%%%%%%%%%%%%%%%%%%%%%%%%%%%%%%%%%%%%%%%%%%%%%%%%%%%%%%%%%%%%%%%%%%%%%%%%%%%%

% Quitar controles de navegación
\usenavigationsymbolstemplate{}

% Numerar las transparencias
\setbeamertemplate{footline}[frame number]

\title{The Computer Scientist as a Toolsmith II}
\subtitle{
  Frederick P. Brooks, Jr. \\
  1994 \\
  \vspace{2em}
  Ingeniería de Software II: \\
  Presentación de Papers Fundacionales
}
\author{
  Sabrina Izcovich \\
  Leandro Lovisolo \\
  Lautaro Petaccio \\
  Sebastián Vita
}
\date{25 de Junio de 2015}
\institute{
  Departamento de Computación \\
  Facultad de Ciencias Exactas y Naturales \\
  Universidad de Buenos Aires
}

\begin{document}

\begin{frame}
  \titlepage
\end{frame}

\begin{frame}
  \begin{center}
 	 \frametitle{Fred Brooks}
    	\begin{itemize}
    		\item Nacido el 19 de abril de 1931 en Estados Unidos.
    		\item Bachiller en Ciencias Físicas, Doctor en Matemáticas Aplicadas (Ciencias de la Computación).
    		\item Encargado de llevar a cabo el desarrollo de la familia del Sistema/360 de IBM.
    		\item Ganó un premio Turing en 1999, entre otros.
    	\end{itemize}
  \end{center}
\end{frame}

\begin{frame}
  \begin{center}
    \Huge{
      ¿Qué es una ciencia?
    }
  \end{center}
\end{frame}

\begin{frame}
  \frametitle{¿Qué es una ciencia?}

  Webster says science is ``a branch of study concerned
  with the observation and classification of facts, especially
  with the establishment and quantitative formulation
  of verifiable general laws.''
\end{frame}

\begin{frame}
  \begin{center}
    \Huge{
      La computación,\\
      ¿es realmente una ciencia?
    }
  \end{center}
\end{frame}

\begin{frame}
  \frametitle{La computación, ¿es realmente una ciencia?}

  Las ``ciencias'' que insisten en llamarse ciencias no lo son.
  \pause
  Ejemplos:
  \pause

  \begin{itemize}
    \item Ciencias sociales \pause
    \item Ciencias económicas \pause
    \item Ciencias políticas \pause
    \item Y claro, ciencias de la computación
  \end{itemize}
\end{frame}

\begin{frame}
  \frametitle{La computación, ¿es realmente una ciencia?}
  \framesubtitle{Ciencias vs. ingenierías}

  \pause

  ¿Cuál es principal distinción entre las ciencias y las ingenierías?

  \pause

  \vspace{2em}
  El propósito.
\end{frame}

\begin{frame}
  \begin{center}
    \Huge{
      Las ciencias construyen \\
      para estudiar.

      \vspace{2em}

      Las ingenierías estudian \\
      para construír.
    }
  \end{center}
\end{frame}

\begin{frame}
  \frametitle{¿Cuál es nuestra disciplina?}

  \pause

  \begin{itemize}
    \item Tesis: nuestra disciplina no es una ciencia, sino una disciplina
      sintética; una ingeniería.
    \pause

    \item Nos preocupamos por \textbf{construir cosas}, sean éstas
      computadoras, algoritmos o sistemas de software.
  \end{itemize}
\end{frame}

\begin{frame}
  \begin{center}
    \Huge{
      El computador científico \\
      como fabricante de herramientas
    }
  \end{center}
\end{frame}

\begin{frame}
  \frametitle{El computador científico como fabricante de herramientas}

  \begin{itemize}
    \item  A diferencia de otras disciplinas como la mecánica o la ingeniería
      civil, las cosas que construimos no son un fin en sí mismo sino
      herramientas para ayudar a otras personas a construir cosas que
      enriquecen la vida humana.

    \pause

    \item Por lo tanto la visión del computador científico como fabricante de
      herramientas es más honesta y adecuada.

    \pause

   \item Nos da un criterio de éxito más adecuado: o bien las herramientas que
     construimos ayudan a los demás a hacer su trabajo más fácil, o no.
  \end{itemize}
\end{frame}

\end{document}
